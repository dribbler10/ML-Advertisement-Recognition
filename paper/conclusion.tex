\section{Заключение}
В работе был дан обзор базовых методов машинного обучения и исследована их эффективность применительно к модельной задаче по распознаванию рекламы. На первом этапе произведен подбор основных параметров методов с целью получения оптимального качества классификации на исходных данных. Как и ожидалось, простые методы (
\(k\)NN, LDA) показали не лучшее качество классификации, но зато они быстро работают. Более продвинутые методы работают медленнее и для них имеет смысл использовать селекцию признаков или эталонов (см. таблицу \ref{table:base-all}). 
\par
Далее были рассмотрены два метода отбора признаков: PCA и отбор по важности с помощью случайного леса (fandom forest selection или RFS). Оба метода оказались бесполезны в сочетании с \(k\)NN и LDA. Применение их вместе с SVM, gradient tree boosting (GTB) и случайным лесом дало схожие результаты: качество классификации SVM незначительно улучшилось (на \(1\%\) и менее), а качество классификации остальных методов до определённого момента слабо зависит от размерности редуцированного пространства признаков. При этом для всех методов, и в особенности для трудоёмких, просматривется линейная зависимость времени обучения от размерности пространства признаков. Таким образом, можно сказать, что редукция размерности имеет смысл для методов SVM, GTB и random forest, причём метод PCA позволяет максимально сократить размерность данных (как правило, до 50 переменных вместо исходных 228) с незначительными потерями качества классификации. Метод RFS в большинстве случаев позволяет оставить около 100 признаков. Его преимуществом является то, что часть признаков можно просто не извлекать при создании новых исходных данных, в то время как для использования PCA это всё равно придётся делать.