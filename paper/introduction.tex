\section{Введение}
Задача распознавания рекламы в видео с развитием вычислительных средств становится всё более доступной. Она является актуальной как для компаний, занимающихся кабельным вещанием или его мониторингом, так и для простых пользователей, желающих записывать и хранить передачи без рекламы.
Одним из имеющихся на рынке средств, позволяющих решать задачу, является редактор Nero Vision, входящий в состав известного пакета Nero Suite. Неизвестно, какой алгоритм при этом используется, но для его работы достаточно обычного домашнего компьютера.
\par
С точки зрения машинного обучения эта задача является задачей бинарной классификации: необходимо пометить каждый кадр видеопотока как содержащий рекламу или нет. В данной работе будет рассмотрена модельная задача распознавания. На её примере планируется рассмотреть работу некоторых методов машинного обучения и техник сокращения объёма данных, требуемых для обучения.